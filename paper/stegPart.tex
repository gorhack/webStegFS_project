\subsection{Steganography}

We began the incorporation of steganography (steg) by researching different techniques, and decided to use the Least Significatn Bit technique due to its ease of implementation. This technique is easily detectable compared to others, but it was deemed acceptable for now. During the research, we found a Git repository that had source code for a Least Significant Bit steg module that matched our requirements. The source code could take a message, embed it into a picture, and then return the new picture. The source code could also take a picture with an embedded message and extract it. Encryption was included in the code, but we removed that functionality in order to maintain simplicity and reliability. In addition, the source code was written for Python 2. Because our application is written in Python 3, we had to modify the code, including what libraries to import, in order to for the steg to be implemented. \smallskip

This first steg module had issues with reliability. In particular, the module would try decoding bytes using utf-8 encoding, and the results would vary. The size of the picture or the message length in characters seemed to have little effect on whether the bytes could be successfully decoded. We could have the decode method catch the error and only return the part of the message that was succesfully read, but this would ultimately yeild unreliable messages. We also tried having a method that would check the image to ensure that the message was properly embedded. This made the steg module cumbersome and very inefficient. The steg module could only be trusted to take very small messages (no greater than 100 characters for most images). We decided to rewrite the steg module in order to ensure reliability.\smallskip

To increase the reliability, we increased the simplicity of the steganography. Instead of trying to manipulate color bit values and
concern ourselves with the encoding and decoding of those bits, we tried manipulating the color decimal values. This steg module was written entirely from scratch by this team. In order to embed a message into an image, a string is concatenated to the end of the message that signals its termination. In this case, "ENDMSG" was added to the end of all messages being embedded. Then, each character of the message is converted to its ASCII decimal value, and then encoded into the color values of each pixel. This method turned out to be very reliable. The main issue was that it was difficult to embed files into images instead of just strings. \smallskip

After the application was tested using this reliable steg module, we decided to go back to using the Least Significatn Bit technique. If we  could write our own module from scratch that manipulated the bits in the picture, we figured it would be easier to place the file bits into the image. By being consistent in only manipulating bits throughout the module, we could avoid the type errors and inefficiences that would arise if we used the steg method described in the previous paragraph. This new steg module is not completed yet and has not been tested. If it works, it would allow us to encode entire directories.