\subsection{Encoding a File System}

The covert file system uses a drop in steganography module that takes in a bytearray object and returns a URL to an image. We used "least significant bit" steganography for this application because of its simplicity and reliability.

\subsubsection{Steganography Implementation}
Least significant bit (LSB) steganography is a Substitution type of steganography~\cite{Nosrati2011} that replaces the least significant bits in the image's pixels. Our steganography module is not true LSB steganography because we replace the least significant bits of each pixel allowing us to encode one byte of data into every pixel. First, we break every byte of the message into three segments. Two segments of the byte will contain three bits, and one segment will contain two bits. Next, we replace the three least significant bits of the red component with the first three bit segment. The green component follows with replacing its least significant bits with the second three bit segment. Lastly, we replace the two least significant bits of the blue component with the remaining two bit segment of data. There are obvious drawbacks to our implementation of LSB steganography, primarily that the image may appear distorted as seen in Figure X. However, with the modularity of the project we used this implementation to increase the amount of data stored while limiting the latency in uploading and downloading images for large file systems.
 
\subsubsection{File System to Images}
The first step in encoding a file system into an image is retrieving an image. We use "The Cat API" which retrieves a random cat picture from Tumblr (c). Once we retrieve the image we determine the number of bytes we can encode in the image by multiplying the height and width of the image, in pixels. Since we encode one byte per pixel, this results in the total number of bytes of data we can encode in the image. Next, we take the data we are going to encode and append a special end of file encoding. If the length of the data is less than the amount of data the image can hold, we encode the data into the image using the steganography method and return the url to the hosted image. Otherwise, we must break the data into multiple segments. In this situation, we must first ... blah blah blah ... encode upload repeat... 

\subsubsection{Steganography and CovertFS}
Based on the requirements of our file system set out in the into steg has obvious risks such as statistical analysis, visible alterations, etc.

Statistical analysis info.

Visible alterations info and pictures.

