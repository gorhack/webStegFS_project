\subsection{File System}

The file system in covertFS is built off of the \textit{pyfilesystem} python package, available on the Python Package Index. \textit{Pyfilesystem}, or \textit{fs}, provides a framework of different modules that can access file systems in different locations- one module accesses the running operating system's file system, there is a module for accessing NFS, and finally there is a module that creates and accesses a file system in main memory. This memory-accessing module is called MemoryFS. 

MemoryFS is the superclass for our custom file system class, named (yes, it is the same as the project name) CovertFS. CovertFS utilizes CovertFiles (subclassed from MemoryFile) to store the data in files, and CovertEntries (subclassed from MemoryEntry) to store the file metadata (MAC times etc.) of all file system entries, including data files and directories. The CovertFS class also adds functionality to work alongside the web connection module of the project, such as the url the corresponding picture can be found at, initial creation time, etcetera. 

In order to mount the user-defined, external, stored-in-memory file system to the native operating system file system, we must use the File system in User Space, or FUSE. FUSE is written in C (to better interact directly with operating system), but there are APIs to use it in other languages--- including Python 3. The python package \textit{fusepy} was our key to accessing FUSE functionality. It provides an easy to use layout for writing a FUSE file system in Python. A FUSE file system is written by defining a function for a list of operations that must be able to be carried out by the operating system--- basically a user definition of how to access each command that can be entered in a command line. Our implementation of this OPERATIONS class is called MemFuse, and it includes a CovertFS object as a parameter. This allows for dynamic variables to be set by the native operating system (through MemFuse), but then later be accessed by the main function, to upload to the online directory of pictures.