% Generated by Sphinx.
\def\sphinxdocclass{report}
\documentclass[letterpaper,10pt,english]{sphinxmanual}
\usepackage[utf8]{inputenc}
\DeclareUnicodeCharacter{00A0}{\nobreakspace}
\usepackage{cmap}
\usepackage[T1]{fontenc}
\usepackage{babel}
\usepackage{times}
\usepackage[Sonny]{fncychap}
\usepackage{longtable}
\usepackage{sphinx}
\usepackage{multirow}


\addto\captionsenglish{\renewcommand{\figurename}{Fig. }}
\addto\captionsenglish{\renewcommand{\tablename}{Table }}
\floatname{literal-block}{Listing }



\title{CovertFS Documentation}
\date{February 26, 2016}
\release{.9}
\author{Kyle Gorak, Adam Sjoholm, David Hart, Ryne Flores}
\newcommand{\sphinxlogo}{}
\renewcommand{\releasename}{Release}
\makeindex

\makeatletter
\def\PYG@reset{\let\PYG@it=\relax \let\PYG@bf=\relax%
    \let\PYG@ul=\relax \let\PYG@tc=\relax%
    \let\PYG@bc=\relax \let\PYG@ff=\relax}
\def\PYG@tok#1{\csname PYG@tok@#1\endcsname}
\def\PYG@toks#1+{\ifx\relax#1\empty\else%
    \PYG@tok{#1}\expandafter\PYG@toks\fi}
\def\PYG@do#1{\PYG@bc{\PYG@tc{\PYG@ul{%
    \PYG@it{\PYG@bf{\PYG@ff{#1}}}}}}}
\def\PYG#1#2{\PYG@reset\PYG@toks#1+\relax+\PYG@do{#2}}

\expandafter\def\csname PYG@tok@nt\endcsname{\let\PYG@bf=\textbf\def\PYG@tc##1{\textcolor[rgb]{0.02,0.16,0.45}{##1}}}
\expandafter\def\csname PYG@tok@nn\endcsname{\let\PYG@bf=\textbf\def\PYG@tc##1{\textcolor[rgb]{0.05,0.52,0.71}{##1}}}
\expandafter\def\csname PYG@tok@sr\endcsname{\def\PYG@tc##1{\textcolor[rgb]{0.14,0.33,0.53}{##1}}}
\expandafter\def\csname PYG@tok@cp\endcsname{\def\PYG@tc##1{\textcolor[rgb]{0.00,0.44,0.13}{##1}}}
\expandafter\def\csname PYG@tok@kr\endcsname{\let\PYG@bf=\textbf\def\PYG@tc##1{\textcolor[rgb]{0.00,0.44,0.13}{##1}}}
\expandafter\def\csname PYG@tok@ow\endcsname{\let\PYG@bf=\textbf\def\PYG@tc##1{\textcolor[rgb]{0.00,0.44,0.13}{##1}}}
\expandafter\def\csname PYG@tok@kt\endcsname{\def\PYG@tc##1{\textcolor[rgb]{0.56,0.13,0.00}{##1}}}
\expandafter\def\csname PYG@tok@gt\endcsname{\def\PYG@tc##1{\textcolor[rgb]{0.00,0.27,0.87}{##1}}}
\expandafter\def\csname PYG@tok@sb\endcsname{\def\PYG@tc##1{\textcolor[rgb]{0.25,0.44,0.63}{##1}}}
\expandafter\def\csname PYG@tok@ne\endcsname{\def\PYG@tc##1{\textcolor[rgb]{0.00,0.44,0.13}{##1}}}
\expandafter\def\csname PYG@tok@bp\endcsname{\def\PYG@tc##1{\textcolor[rgb]{0.00,0.44,0.13}{##1}}}
\expandafter\def\csname PYG@tok@mf\endcsname{\def\PYG@tc##1{\textcolor[rgb]{0.13,0.50,0.31}{##1}}}
\expandafter\def\csname PYG@tok@gs\endcsname{\let\PYG@bf=\textbf}
\expandafter\def\csname PYG@tok@kc\endcsname{\let\PYG@bf=\textbf\def\PYG@tc##1{\textcolor[rgb]{0.00,0.44,0.13}{##1}}}
\expandafter\def\csname PYG@tok@s2\endcsname{\def\PYG@tc##1{\textcolor[rgb]{0.25,0.44,0.63}{##1}}}
\expandafter\def\csname PYG@tok@il\endcsname{\def\PYG@tc##1{\textcolor[rgb]{0.13,0.50,0.31}{##1}}}
\expandafter\def\csname PYG@tok@gh\endcsname{\let\PYG@bf=\textbf\def\PYG@tc##1{\textcolor[rgb]{0.00,0.00,0.50}{##1}}}
\expandafter\def\csname PYG@tok@ss\endcsname{\def\PYG@tc##1{\textcolor[rgb]{0.32,0.47,0.09}{##1}}}
\expandafter\def\csname PYG@tok@k\endcsname{\let\PYG@bf=\textbf\def\PYG@tc##1{\textcolor[rgb]{0.00,0.44,0.13}{##1}}}
\expandafter\def\csname PYG@tok@c1\endcsname{\let\PYG@it=\textit\def\PYG@tc##1{\textcolor[rgb]{0.25,0.50,0.56}{##1}}}
\expandafter\def\csname PYG@tok@mo\endcsname{\def\PYG@tc##1{\textcolor[rgb]{0.13,0.50,0.31}{##1}}}
\expandafter\def\csname PYG@tok@ni\endcsname{\let\PYG@bf=\textbf\def\PYG@tc##1{\textcolor[rgb]{0.84,0.33,0.22}{##1}}}
\expandafter\def\csname PYG@tok@gi\endcsname{\def\PYG@tc##1{\textcolor[rgb]{0.00,0.63,0.00}{##1}}}
\expandafter\def\csname PYG@tok@kd\endcsname{\let\PYG@bf=\textbf\def\PYG@tc##1{\textcolor[rgb]{0.00,0.44,0.13}{##1}}}
\expandafter\def\csname PYG@tok@nc\endcsname{\let\PYG@bf=\textbf\def\PYG@tc##1{\textcolor[rgb]{0.05,0.52,0.71}{##1}}}
\expandafter\def\csname PYG@tok@nf\endcsname{\def\PYG@tc##1{\textcolor[rgb]{0.02,0.16,0.49}{##1}}}
\expandafter\def\csname PYG@tok@ge\endcsname{\let\PYG@it=\textit}
\expandafter\def\csname PYG@tok@gp\endcsname{\let\PYG@bf=\textbf\def\PYG@tc##1{\textcolor[rgb]{0.78,0.36,0.04}{##1}}}
\expandafter\def\csname PYG@tok@mi\endcsname{\def\PYG@tc##1{\textcolor[rgb]{0.13,0.50,0.31}{##1}}}
\expandafter\def\csname PYG@tok@nb\endcsname{\def\PYG@tc##1{\textcolor[rgb]{0.00,0.44,0.13}{##1}}}
\expandafter\def\csname PYG@tok@sd\endcsname{\let\PYG@it=\textit\def\PYG@tc##1{\textcolor[rgb]{0.25,0.44,0.63}{##1}}}
\expandafter\def\csname PYG@tok@nl\endcsname{\let\PYG@bf=\textbf\def\PYG@tc##1{\textcolor[rgb]{0.00,0.13,0.44}{##1}}}
\expandafter\def\csname PYG@tok@err\endcsname{\def\PYG@bc##1{\setlength{\fboxsep}{0pt}\fcolorbox[rgb]{1.00,0.00,0.00}{1,1,1}{\strut ##1}}}
\expandafter\def\csname PYG@tok@mh\endcsname{\def\PYG@tc##1{\textcolor[rgb]{0.13,0.50,0.31}{##1}}}
\expandafter\def\csname PYG@tok@s1\endcsname{\def\PYG@tc##1{\textcolor[rgb]{0.25,0.44,0.63}{##1}}}
\expandafter\def\csname PYG@tok@si\endcsname{\let\PYG@it=\textit\def\PYG@tc##1{\textcolor[rgb]{0.44,0.63,0.82}{##1}}}
\expandafter\def\csname PYG@tok@sc\endcsname{\def\PYG@tc##1{\textcolor[rgb]{0.25,0.44,0.63}{##1}}}
\expandafter\def\csname PYG@tok@sh\endcsname{\def\PYG@tc##1{\textcolor[rgb]{0.25,0.44,0.63}{##1}}}
\expandafter\def\csname PYG@tok@no\endcsname{\def\PYG@tc##1{\textcolor[rgb]{0.38,0.68,0.84}{##1}}}
\expandafter\def\csname PYG@tok@gr\endcsname{\def\PYG@tc##1{\textcolor[rgb]{1.00,0.00,0.00}{##1}}}
\expandafter\def\csname PYG@tok@kn\endcsname{\let\PYG@bf=\textbf\def\PYG@tc##1{\textcolor[rgb]{0.00,0.44,0.13}{##1}}}
\expandafter\def\csname PYG@tok@m\endcsname{\def\PYG@tc##1{\textcolor[rgb]{0.13,0.50,0.31}{##1}}}
\expandafter\def\csname PYG@tok@se\endcsname{\let\PYG@bf=\textbf\def\PYG@tc##1{\textcolor[rgb]{0.25,0.44,0.63}{##1}}}
\expandafter\def\csname PYG@tok@cs\endcsname{\def\PYG@tc##1{\textcolor[rgb]{0.25,0.50,0.56}{##1}}\def\PYG@bc##1{\setlength{\fboxsep}{0pt}\colorbox[rgb]{1.00,0.94,0.94}{\strut ##1}}}
\expandafter\def\csname PYG@tok@nd\endcsname{\let\PYG@bf=\textbf\def\PYG@tc##1{\textcolor[rgb]{0.33,0.33,0.33}{##1}}}
\expandafter\def\csname PYG@tok@nv\endcsname{\def\PYG@tc##1{\textcolor[rgb]{0.73,0.38,0.84}{##1}}}
\expandafter\def\csname PYG@tok@gu\endcsname{\let\PYG@bf=\textbf\def\PYG@tc##1{\textcolor[rgb]{0.50,0.00,0.50}{##1}}}
\expandafter\def\csname PYG@tok@vi\endcsname{\def\PYG@tc##1{\textcolor[rgb]{0.73,0.38,0.84}{##1}}}
\expandafter\def\csname PYG@tok@vc\endcsname{\def\PYG@tc##1{\textcolor[rgb]{0.73,0.38,0.84}{##1}}}
\expandafter\def\csname PYG@tok@vg\endcsname{\def\PYG@tc##1{\textcolor[rgb]{0.73,0.38,0.84}{##1}}}
\expandafter\def\csname PYG@tok@gd\endcsname{\def\PYG@tc##1{\textcolor[rgb]{0.63,0.00,0.00}{##1}}}
\expandafter\def\csname PYG@tok@kp\endcsname{\def\PYG@tc##1{\textcolor[rgb]{0.00,0.44,0.13}{##1}}}
\expandafter\def\csname PYG@tok@sx\endcsname{\def\PYG@tc##1{\textcolor[rgb]{0.78,0.36,0.04}{##1}}}
\expandafter\def\csname PYG@tok@cm\endcsname{\let\PYG@it=\textit\def\PYG@tc##1{\textcolor[rgb]{0.25,0.50,0.56}{##1}}}
\expandafter\def\csname PYG@tok@o\endcsname{\def\PYG@tc##1{\textcolor[rgb]{0.40,0.40,0.40}{##1}}}
\expandafter\def\csname PYG@tok@mb\endcsname{\def\PYG@tc##1{\textcolor[rgb]{0.13,0.50,0.31}{##1}}}
\expandafter\def\csname PYG@tok@w\endcsname{\def\PYG@tc##1{\textcolor[rgb]{0.73,0.73,0.73}{##1}}}
\expandafter\def\csname PYG@tok@go\endcsname{\def\PYG@tc##1{\textcolor[rgb]{0.20,0.20,0.20}{##1}}}
\expandafter\def\csname PYG@tok@c\endcsname{\let\PYG@it=\textit\def\PYG@tc##1{\textcolor[rgb]{0.25,0.50,0.56}{##1}}}
\expandafter\def\csname PYG@tok@s\endcsname{\def\PYG@tc##1{\textcolor[rgb]{0.25,0.44,0.63}{##1}}}
\expandafter\def\csname PYG@tok@na\endcsname{\def\PYG@tc##1{\textcolor[rgb]{0.25,0.44,0.63}{##1}}}

\def\PYGZbs{\char`\\}
\def\PYGZus{\char`\_}
\def\PYGZob{\char`\{}
\def\PYGZcb{\char`\}}
\def\PYGZca{\char`\^}
\def\PYGZam{\char`\&}
\def\PYGZlt{\char`\<}
\def\PYGZgt{\char`\>}
\def\PYGZsh{\char`\#}
\def\PYGZpc{\char`\%}
\def\PYGZdl{\char`\$}
\def\PYGZhy{\char`\-}
\def\PYGZsq{\char`\'}
\def\PYGZdq{\char`\"}
\def\PYGZti{\char`\~}
% for compatibility with earlier versions
\def\PYGZat{@}
\def\PYGZlb{[}
\def\PYGZrb{]}
\makeatother

\renewcommand\PYGZsq{\textquotesingle}

\begin{document}

\maketitle
\tableofcontents
\phantomsection\label{index::doc}


\chapter{About the Project}
\label{index:about-the-project}
Covert File System is a web-based application that allows users to covertly share files through social media sites while maintaining plausible deniability for both the user(s) and the social media site.

We created this project for a Capstone class for Computer Science at the United States Military Academy, West Point. This project was broken up into the following sprints over the course of one year.


\section{Sprint 1: Knowledge acquisition}
\label{index:sprint-1-knowledge-acquisition}
Sprint broken into three sub-goals:
\begin{enumerate}
\item {} 
Implement basic steg module to encode and decode an image in Python
\begin{itemize}
\item {} 
\code{\$ python3 Image\_Manipulation/lsbsteg.py {[}encode/decode{]} -i {[}image\_path{]} -m '{[}message{]}'}

\item {} 
encode saves a copy of the image\_path with a `\_1' appended encoded with the message

\item {} 
decode prints the encoded message in the image\_path

\end{itemize}

\item {} 
Determine a suitable social media site that meets our requirements (anonymous user upload, no or lossless compression)
\begin{itemize}
\item {} 
social media sites investigated:
- SendSpace
- Whisper
- Flickr
- Yogile

\end{itemize}

\item {} 
Design and implement basic upload application in Python for the selected social media site
\begin{itemize}
\item {} 
\code{\$ python3 Web\_Connection/api\_cons.py}

\item {} 
returns the download url for the uploaded random cat image (stores the delete URL as well)

\end{itemize}

\end{enumerate}


\section{Sprint 2: Covert Mapping Structure}
\label{index:sprint-2-covert-mapping-structure}
Sprint broken into four sub-goals
\begin{enumerate}
\item {} 
Design the map structure for the covert file system to allow maximum flexibility and usability.

\item {} 
Break a large message file into parts to encode across multiple images.
\begin{itemize}
\item {} 
Analysis of how much data can be encoded using LSB

\item {} 
Determine file system overhead in each image

\end{itemize}

\item {} 
Begin to add API connection and Encode/Decoder into Application.
\begin{itemize}
\item {} 
\code{\$ python3 main.py {[}url{]}} \footnote[1]{
Optional parameter
} url can be the full url path or the file id (6 character ending, i.e xvdmcn)

\item {} 
\code{covertFS\$ {[}command{]}}

\end{itemize}

\item {} 
Functional Design Documents

\item {} 
From previous Sprint:
\begin{itemize}
\item {} 
Keep all images in memory

\item {} 
Error handling in API connection

\item {} 
Enforce restrictions on arguments in encode/decode

\end{itemize}

\end{enumerate}


\section{Sprint 3: Beta release}
\label{index:sprint-3-beta-release}
Basic stand-alone application to encode/decode a local covert file-system that is able to store, open, and delete files from the covert file-system. Command line program will work similar to a unix based directory system. Using these commands will require breaking the file structure across multiple encoded images. Everything is seamless to the user who only needs to keep track of the /root image URL and then navigate the file system with ease.


\section{Sprint 4: Publication start and alpha release}
\label{index:sprint-4-publication-start-and-alpha-release}
Sprint broken down into 5 sub-goals:
\begin{enumerate}
\item {} 
Basic draft of paper for publication using LaTEX.

\item {} 
Create a backlog of things required to implement covertFS into a live operating system such as Tails.

\item {} 
Publish documentation using apidocs.

\item {} 
Create a FUSE module for covertFS.

\item {} 
Change steg technique to allow storage of larger files with dynamic sizes.

\end{enumerate}


\section{Sprint 5: Publication draft and beta release}
\label{index:sprint-5-publication-draft-and-beta-release}
Sprint broken down into 6 sub-goals:
\begin{enumerate}
\item {} 
Encode/decode any file

\item {} 
Background process/thread for uploading and downloading images

\item {} 
Modularize classes

\item {} 
Rough draft (80\%) publication

\item {} 
Working implementation of covertFS on Tails OS

\item {} 
Modular encryption class

\end{enumerate}


\chapter{Setup}
\label{index:setup}\begin{itemize}
\item {} 
Clone the project from GitHub \code{git clone https://github.com/gorhack/covertFS.git}

\end{itemize}

Covert File System is written exclusively in Python 3 due to the vast modules and libraries that support our goals. Currently \emph{covertFS} only supports using \emph{sendspace} for upload on the web.
\begin{itemize}
\item {} 
Dependencies:
\begin{itemize}
\item {} 
Install \href{https://pillow.readthedocs.org/en/3.0.0/installation.html}{Pillow} dependecies

\item {} 
python3, pip3 \code{\$ pip3 install -r utls/requirements.txt}

\item {} 
On MacOSX install \href{https://osxfuse.github.io}{FUSE}

\item {} 
Get a sendspace API key \href{https://www.sendspace.com/dev\_apikeys.html}{here}

\item {} 
Copy your sendspace API key and create a file in /Web\_Connection/API\_Keys/ containing \code{sendSpaceKey='API KEY GOES HERE'}

\item {} 
\code{\$ sudo python3 setup.py install}

\end{itemize}

\end{itemize}


\chapter{Usage:}
\label{index:usage}\begin{itemize}
\item {} 
\code{\$ python3 src/main.py {[}url to fs image{]} {[}-c{]} {[}-w{]} {[}-p{]} {[}-e{]} {[}-m{]} {[}-s{]}} \footnotemark[1]
\begin{itemize}
\item {} 
-c command loop, run covertFS shell

\item {} 
-w social media site to upload/download images {[}default: sendspace{]}

\item {} 
-p use built in proxy

\item {} 
-e encryption type

\item {} 
-m mount point {[}default: covertMount{]}

\item {} 
-s steganography {[}default: LSBsteg{]}

\end{itemize}

\item {} 
\code{covertFS\$ {[}command{]}} basic application usage.
\begin{itemize}
\item {} 
\code{mount} mounts the file system using FUSE

\item {} 
\code{proxy} / \code{noproxy} turns the built in proxy on/off respectively

\item {} 
\code{loadfs {[}url{]}} load a covert file system

\item {} 
\code{newfs} uploads the old fs and returns the url. loads a new covert file system.

\item {} 
\code{encodeimage {[}file{]}} encode an image with a file, returns the URL to the image

\item {} 
\code{decodeimage {[}url{]}} decode an image, returns the file

\item {} 
\code{uploadfs {[}url{]}} save the covert file system, returns URL to the root image. To load the same file system this URL must be retained.

\item {} 
\code{mkdir {[}path{]}} make directories in the covert file system at the given path

\item {} 
\code{rmdir {[}path{]}} remove directories in the covert file system at the given path

\item {} 
\code{mkfile {[}name{]} {[}text{]} {[}path{]}} create a text file in the covert file system at the path

\item {} 
\code{upload {[}local path{]} {[}covert path{]}} upload a file to the covert file system

\item {} 
\code{download {[}covert path{]} {[}local path{]}} download a file on the covert file system to disk

\item {} 
\code{ls {[}path{]}} \footnotemark[1] list directory contents

\item {} 
\code{cd {[}path{]}} change directory in the covert file system to the path

\item {} 
\code{cat {[}file{]}} concatenate and print files

\item {} 
\code{rm {[}path{]}} remove a file from the covert file system

\item {} 
\code{hist} show the history of previous commands

\item {} 
\code{shell {[}cmd{]}} run shell commands

\item {} 
\code{help {[}cmd{]}} \footnotemark[1] show list of commands or documentation for a specific command

\item {} 
\code{exit} exit the covert file system

\end{itemize}

\end{itemize}


\chapter{Source Documentation}
\label{index:source-documentation}

\section{Command line main module}
\label{main:command-line-main-module}\label{main::doc}
\emph{covertFS} is a command line based program created using the \emph{cmd} module. The \emph{main.py} file contains all commands available and additional helper functions.
\phantomsection\label{main:module-main}\index{main (module)}\index{Console (class in main)}

\begin{fulllineitems}
\phantomsection\label{main:main.Console}\pysiglinewithargsret{\strong{class }\code{main.}\bfcode{Console}}{\emph{online\_file\_store}, \emph{steg\_class}, \emph{mountpoint}, \emph{url}, \emph{proxy}, \emph{cmdLoopUsed}}{}
Bases: \code{cmd.Cmd}, \code{object}
\index{add\_file\_to\_fs() (main.Console method)}

\begin{fulllineitems}
\phantomsection\label{main:main.Console.add_file_to_fs}\pysiglinewithargsret{\bfcode{add\_file\_to\_fs}}{\emph{fspath}, \emph{contents}}{}
Helper function to add a file to the fs.

\end{fulllineitems}

\index{completedefault() (main.Console method)}

\begin{fulllineitems}
\phantomsection\label{main:main.Console.completedefault}\pysiglinewithargsret{\bfcode{completedefault}}{\emph{text}, \emph{line}, \emph{begidx}, \emph{endidx}}{}
Allow Tab autocompletion of file names.

\end{fulllineitems}

\index{default() (main.Console method)}

\begin{fulllineitems}
\phantomsection\label{main:main.Console.default}\pysiglinewithargsret{\bfcode{default}}{\emph{line}}{}
Called on an input line when the command prefix is not recognized.
In that case we execute the line as Python code.

\end{fulllineitems}

\index{do\_EOF() (main.Console method)}

\begin{fulllineitems}
\phantomsection\label{main:main.Console.do_EOF}\pysiglinewithargsret{\bfcode{do\_EOF}}{\emph{args}}{}
Exit on system end of file character

\end{fulllineitems}

\index{do\_cat() (main.Console method)}

\begin{fulllineitems}
\phantomsection\label{main:main.Console.do_cat}\pysiglinewithargsret{\bfcode{do\_cat}}{\emph{args}}{}
View the contents of a file in the file system.

Use: cat {[}covert path{]}

\end{fulllineitems}

\index{do\_cd() (main.Console method)}

\begin{fulllineitems}
\phantomsection\label{main:main.Console.do_cd}\pysiglinewithargsret{\bfcode{do\_cd}}{\emph{args}}{}
Change directory to specified {[}path{]}

Use: cd {[}path{]}*

\end{fulllineitems}

\index{do\_decodeimage() (main.Console method)}

\begin{fulllineitems}
\phantomsection\label{main:main.Console.do_decodeimage}\pysiglinewithargsret{\bfcode{do\_decodeimage}}{\emph{file\_id}}{}
Decode the message in an image.

Returns the message in plain text.

decodeimage {[}download url{]}

\end{fulllineitems}

\index{do\_download() (main.Console method)}

\begin{fulllineitems}
\phantomsection\label{main:main.Console.do_download}\pysiglinewithargsret{\bfcode{do\_download}}{\emph{args}}{}
Download a covert file to the local file system.

Use: download {[}covert path{]} {[}local path{]}

\end{fulllineitems}

\index{do\_encodeimage() (main.Console method)}

\begin{fulllineitems}
\phantomsection\label{main:main.Console.do_encodeimage}\pysiglinewithargsret{\bfcode{do\_encodeimage}}{\emph{file}}{}
Encode a file and upload to social media.

Returns the url.

Use: encodeimage {[}file{]}

\end{fulllineitems}

\index{do\_exit() (main.Console method)}

\begin{fulllineitems}
\phantomsection\label{main:main.Console.do_exit}\pysiglinewithargsret{\bfcode{do\_exit}}{\emph{args}}{}
Exits from the console

\end{fulllineitems}

\index{do\_help() (main.Console method)}

\begin{fulllineitems}
\phantomsection\label{main:main.Console.do_help}\pysiglinewithargsret{\bfcode{do\_help}}{\emph{args}}{}~\begin{description}
\item[{Get help on commands}] \leavevmode
`help' or `?' with no arguments prints the list of commands
`help \textless{}command\textgreater{}' or `? \textless{}command\textgreater{}' gives help on \textless{}command\textgreater{}

\end{description}

\end{fulllineitems}

\index{do\_hist() (main.Console method)}

\begin{fulllineitems}
\phantomsection\label{main:main.Console.do_hist}\pysiglinewithargsret{\bfcode{do\_hist}}{\emph{args}}{}
Print a list of commands that have been entered

\end{fulllineitems}

\index{do\_loadfs() (main.Console method)}

\begin{fulllineitems}
\phantomsection\label{main:main.Console.do_loadfs}\pysiglinewithargsret{\bfcode{do\_loadfs}}{\emph{url}}{}
Load a covert file system.

Use: loadfs {[}url{]}

\end{fulllineitems}

\index{do\_ls() (main.Console method)}

\begin{fulllineitems}
\phantomsection\label{main:main.Console.do_ls}\pysiglinewithargsret{\bfcode{do\_ls}}{\emph{args}}{}
List items in directory

Use: ls {[}path{]}*

\end{fulllineitems}

\index{do\_mkdir() (main.Console method)}

\begin{fulllineitems}
\phantomsection\label{main:main.Console.do_mkdir}\pysiglinewithargsret{\bfcode{do\_mkdir}}{\emph{args}}{}
Make a folder at the given path.

Use: mkdir {[}path{]}

\end{fulllineitems}

\index{do\_mkfile() (main.Console method)}

\begin{fulllineitems}
\phantomsection\label{main:main.Console.do_mkfile}\pysiglinewithargsret{\bfcode{do\_mkfile}}{\emph{args}}{}
Create a text file with a message to the file system.

Use: mkfile {[}path{]} {[}message{]}

\end{fulllineitems}

\index{do\_mount() (main.Console method)}

\begin{fulllineitems}
\phantomsection\label{main:main.Console.do_mount}\pysiglinewithargsret{\bfcode{do\_mount}}{\emph{args}}{}
\end{fulllineitems}

\index{do\_newfs() (main.Console method)}

\begin{fulllineitems}
\phantomsection\label{main:main.Console.do_newfs}\pysiglinewithargsret{\bfcode{do\_newfs}}{\emph{args}}{}
Create a covert file system, return the URL of the old fs.

Use: newfs

\end{fulllineitems}

\index{do\_noproxy() (main.Console method)}

\begin{fulllineitems}
\phantomsection\label{main:main.Console.do_noproxy}\pysiglinewithargsret{\bfcode{do\_noproxy}}{\emph{args}}{}
Turns off the built-in proxy.

Use: noproxy

\end{fulllineitems}

\index{do\_proxy() (main.Console method)}

\begin{fulllineitems}
\phantomsection\label{main:main.Console.do_proxy}\pysiglinewithargsret{\bfcode{do\_proxy}}{\emph{args}}{}
Turns on the built-in proxy.

Use: proxy

\end{fulllineitems}

\index{do\_rm() (main.Console method)}

\begin{fulllineitems}
\phantomsection\label{main:main.Console.do_rm}\pysiglinewithargsret{\bfcode{do\_rm}}{\emph{args}}{}
Remove a file from the covert file system.

Use: rm {[}path{]}*

\end{fulllineitems}

\index{do\_rmdir() (main.Console method)}

\begin{fulllineitems}
\phantomsection\label{main:main.Console.do_rmdir}\pysiglinewithargsret{\bfcode{do\_rmdir}}{\emph{args}}{}
Remove a folder in the current directory.

Use: rmdir {[}path{]}

\end{fulllineitems}

\index{do\_shell() (main.Console method)}

\begin{fulllineitems}
\phantomsection\label{main:main.Console.do_shell}\pysiglinewithargsret{\bfcode{do\_shell}}{\emph{args}}{}
Pass command to a system shell when line begins with `!'

\end{fulllineitems}

\index{do\_upload() (main.Console method)}

\begin{fulllineitems}
\phantomsection\label{main:main.Console.do_upload}\pysiglinewithargsret{\bfcode{do\_upload}}{\emph{args}}{}
Upload a local file to the covert file system.

Use: upload {[}local path{]} {[}covert path{]}

\end{fulllineitems}

\index{do\_uploadfs() (main.Console method)}

\begin{fulllineitems}
\phantomsection\label{main:main.Console.do_uploadfs}\pysiglinewithargsret{\bfcode{do\_uploadfs}}{\emph{args}}{}
Upload covert fileSystem to the web

\end{fulllineitems}

\index{down\_and\_set\_file() (main.Console method)}

\begin{fulllineitems}
\phantomsection\label{main:main.Console.down_and_set_file}\pysiglinewithargsret{\bfcode{down\_and\_set\_file}}{\emph{filename}}{}
Download a file. Put it in the filesystem.

\end{fulllineitems}

\index{emptyline() (main.Console method)}

\begin{fulllineitems}
\phantomsection\label{main:main.Console.emptyline}\pysiglinewithargsret{\bfcode{emptyline}}{}{}
Do nothing on empty input line

\end{fulllineitems}

\index{init\_factory() (main.Console method)}

\begin{fulllineitems}
\phantomsection\label{main:main.Console.init_factory}\pysiglinewithargsret{\bfcode{init\_factory}}{}{}
\end{fulllineitems}

\index{loadfs() (main.Console method)}

\begin{fulllineitems}
\phantomsection\label{main:main.Console.loadfs}\pysiglinewithargsret{\bfcode{loadfs}}{}{}
Load the filesystem from a URL: Download pic, decode it, then send the string to the load function in fsClass.

\end{fulllineitems}

\index{open\_window() (main.Console method)}

\begin{fulllineitems}
\phantomsection\label{main:main.Console.open_window}\pysiglinewithargsret{\bfcode{open\_window}}{}{}
\end{fulllineitems}

\index{postcmd() (main.Console method)}

\begin{fulllineitems}
\phantomsection\label{main:main.Console.postcmd}\pysiglinewithargsret{\bfcode{postcmd}}{\emph{stop}, \emph{line}}{}
If you want to stop the console, return something that evaluates to true.
If you want to do some post command processing, do it here.

\end{fulllineitems}

\index{postloop() (main.Console method)}

\begin{fulllineitems}
\phantomsection\label{main:main.Console.postloop}\pysiglinewithargsret{\bfcode{postloop}}{}{}
Take care of any unfinished business.
Despite the claims in the Cmd documentaion,
Cmd.postloop() is not a stub.

\end{fulllineitems}

\index{precmd() (main.Console method)}

\begin{fulllineitems}
\phantomsection\label{main:main.Console.precmd}\pysiglinewithargsret{\bfcode{precmd}}{\emph{line}}{}
This method is called after the line has been input but before
it has been interpreted. If you want to modifdy the input line
before execution (for example, variable substitution) do it here.

\end{fulllineitems}

\index{preloop() (main.Console method)}

\begin{fulllineitems}
\phantomsection\label{main:main.Console.preloop}\pysiglinewithargsret{\bfcode{preloop}}{}{}
Initialization before prompting user for commands.
Despite the claims in the Cmd documentaion,
Cmd.preloop() is not a stub.

\end{fulllineitems}

\index{san\_file() (main.Console method)}

\begin{fulllineitems}
\phantomsection\label{main:main.Console.san_file}\pysiglinewithargsret{\bfcode{san\_file}}{\emph{file\_contents}}{}
Sanitize file before 1)viewing contents or 2)putting on host OS

\end{fulllineitems}

\index{upload\_file() (main.Console method)}

\begin{fulllineitems}
\phantomsection\label{main:main.Console.upload_file}\pysiglinewithargsret{\bfcode{upload\_file}}{\emph{contents}}{}
Helper function to upload file, return the download url.

\end{fulllineitems}


\end{fulllineitems}

\index{proxy\_parser() (in module main)}

\begin{fulllineitems}
\phantomsection\label{main:main.proxy_parser}\pysiglinewithargsret{\code{main.}\bfcode{proxy\_parser}}{\emph{proxyString=None}}{}
\end{fulllineitems}

\index{proxy\_test() (in module main)}

\begin{fulllineitems}
\phantomsection\label{main:main.proxy_test}\pysiglinewithargsret{\code{main.}\bfcode{proxy\_test}}{\emph{proxyL}}{}
\end{fulllineitems}



\section{File\_System package}
\label{File_System:file-system-package}\label{File_System::doc}

\subsection{Submodules}
\label{File_System:submodules}

\subsubsection{covertfs module}
\label{File_System:covertfs-module}
The \emph{covertfs} module extends the \emph{pyfilesystem} package, using the \emph{MemoryFS} file system.
The MemoryFS file system stores all directory and file info in main memory, to allow for instantaneous file access as well as to avoid writing any FS information to disk. This allows for plausible deniability. All filesystem-necessary commands (ls, cd, mkdir, rm etc) are extended in this module.

The covertfs module includes CovertFile, a subclass of MemoryFile, CovertEntry, a subclass of MemoryEntry, and CovertFS, a subclass of MemoryFS. CovertFS uses CovertFile as the file factory, and CovertEntry as the entry factory.
\phantomsection\label{File_System:module-File_System.covertfs}\index{File\_System.covertfs (module)}\index{CovertEntry (class in File\_System.covertfs)}

\begin{fulllineitems}
\phantomsection\label{File_System:File_System.covertfs.CovertEntry}\pysiglinewithargsret{\strong{class }\code{File\_System.covertfs.}\bfcode{CovertEntry}}{\emph{type}, \emph{name}, \emph{contents=None}}{}
Bases: \code{File\_System.memoryfs.DirEntry}

\end{fulllineitems}

\index{CovertFS (class in File\_System.covertfs)}

\begin{fulllineitems}
\phantomsection\label{File_System:File_System.covertfs.CovertFS}\pysigline{\strong{class }\code{File\_System.covertfs.}\bfcode{CovertFS}}
Bases: \code{File\_System.memoryfs.MemoryFS}
\index{addfile() (File\_System.covertfs.CovertFS method)}

\begin{fulllineitems}
\phantomsection\label{File_System:File_System.covertfs.CovertFS.addfile}\pysiglinewithargsret{\bfcode{addfile}}{\emph{path}, \emph{contents}}{}
Add a file, with given contents, to given path.
Error if path is a file, directory, or if parent directory
is not present.

\end{fulllineitems}

\index{cd() (File\_System.covertfs.CovertFS method)}

\begin{fulllineitems}
\phantomsection\label{File_System:File_System.covertfs.CovertFS.cd}\pysiglinewithargsret{\bfcode{cd}}{\emph{path='/'}}{}
Changes current directory. Superclass has no concept of
current directory (all calls are made from root dir), so
this method is purely local.
Error if given path does not exist, or is a file.

\end{fulllineitems}

\index{check\_parent\_dir() (File\_System.covertfs.CovertFS method)}

\begin{fulllineitems}
\phantomsection\label{File_System:File_System.covertfs.CovertFS.check_parent_dir}\pysiglinewithargsret{\bfcode{check\_parent\_dir}}{\emph{path}}{}
Checks to ensure parent directory is present before
attempting to add a file to it.

\end{fulllineitems}

\index{loadfs() (File\_System.covertfs.CovertFS method)}

\begin{fulllineitems}
\phantomsection\label{File_System:File_System.covertfs.CovertFS.loadfs}\pysiglinewithargsret{\bfcode{loadfs}}{\emph{fsstring}}{}
Iterates through a string that represents the filesystem
(either pulled from online, or given as a test string),
makes necessary directories, and creates necessary files
(empty for now) that are then loaded by main.py.

\end{fulllineitems}

\index{ls() (File\_System.covertfs.CovertFS method)}

\begin{fulllineitems}
\phantomsection\label{File_System:File_System.covertfs.CovertFS.ls}\pysiglinewithargsret{\bfcode{ls}}{\emph{path=None}}{}
Returns a list of the files and directories in the given
path, or the current directory if no path is given.
Error if given path does not exist, or is a file.

\end{fulllineitems}

\index{mkdir() (File\_System.covertfs.CovertFS method)}

\begin{fulllineitems}
\phantomsection\label{File_System:File_System.covertfs.CovertFS.mkdir}\pysiglinewithargsret{\bfcode{mkdir}}{\emph{path}}{}
Makes a new directory at given path.
Error if path is a directory already, or a file.

\end{fulllineitems}

\index{rm() (File\_System.covertfs.CovertFS method)}

\begin{fulllineitems}
\phantomsection\label{File_System:File_System.covertfs.CovertFS.rm}\pysiglinewithargsret{\bfcode{rm}}{\emph{path}}{}
Removes file at given path.
Error if no such file.

\end{fulllineitems}

\index{rmdir() (File\_System.covertfs.CovertFS method)}

\begin{fulllineitems}
\phantomsection\label{File_System:File_System.covertfs.CovertFS.rmdir}\pysiglinewithargsret{\bfcode{rmdir}}{\emph{path=None}, \emph{force=False}}{}
Removes empty directory at given path, or non-empty
directory if force option is given.
Error if path is not a directory.

\end{fulllineitems}

\index{sanitize\_path() (File\_System.covertfs.CovertFS method)}

\begin{fulllineitems}
\phantomsection\label{File_System:File_System.covertfs.CovertFS.sanitize_path}\pysiglinewithargsret{\bfcode{sanitize\_path}}{\emph{path=None}}{}
This method takes a user-input path and makes it
method-readable, by adding the current path to the
front (if the desired path doesn't start with /) or
returning the root path if no path is given.

\end{fulllineitems}

\index{save() (File\_System.covertfs.CovertFS method)}

\begin{fulllineitems}
\phantomsection\label{File_System:File_System.covertfs.CovertFS.save}\pysiglinewithargsret{\bfcode{save}}{}{}
Turns the entire filesystem into a string to be uploaded.
Returns that string.

\end{fulllineitems}

\index{setcreate() (File\_System.covertfs.CovertFS method)}

\begin{fulllineitems}
\phantomsection\label{File_System:File_System.covertfs.CovertFS.setcreate}\pysiglinewithargsret{\bfcode{setcreate}}{\emph{path}, \emph{timeIn=None}}{}
Sets the creation time of a file or directory. Used when the filesystem is loaded from online repository

\end{fulllineitems}

\index{settimes() (File\_System.covertfs.CovertFS method)}

\begin{fulllineitems}
\phantomsection\label{File_System:File_System.covertfs.CovertFS.settimes}\pysiglinewithargsret{\bfcode{settimes}}{\emph{path}, \emph{accessed\_time=None}, \emph{modified\_time=None}}{}
Sets accessed and modified times of a file or directory after file operations take place

\end{fulllineitems}


\end{fulllineitems}

\index{CovertFile (class in File\_System.covertfs)}

\begin{fulllineitems}
\phantomsection\label{File_System:File_System.covertfs.CovertFile}\pysiglinewithargsret{\strong{class }\code{File\_System.covertfs.}\bfcode{CovertFile}}{\emph{path}, \emph{memory\_fs}, \emph{mem\_file}, \emph{mode}, \emph{lock}}{}
Bases: \code{File\_System.memoryfs.MemoryFile}

\end{fulllineitems}



\subsubsection{memfuse module}
\label{File_System:memfuse-module}
The \emph{memfuse} module extends the \emph{pyfuse} package, specifically the \emph{Operations} module.
`Operations' is linked to the Unix FUSE package with python ctypes, to make the file system available to the user, without making system calls. The memfuse module extends all necessary file system operations in a way that is accessible by FUSE, effectively linking the CovertFS class to the FUSE package. When mounted (using FUSE) onto the native Linux file system, memfuse provides access to all files and directories in the CovertFS file system.
\phantomsection\label{File_System:module-File_System.memfuse}\index{File\_System.memfuse (module)}\index{MemFS (class in File\_System.memfuse)}

\begin{fulllineitems}
\phantomsection\label{File_System:File_System.memfuse.MemFS}\pysiglinewithargsret{\strong{class }\code{File\_System.memfuse.}\bfcode{MemFS}}{\emph{memoryfs}}{}
Bases: \code{File\_System.fuse.LoggingMixIn}, \code{File\_System.fuse.Operations}

Subclass of operations. Links all necessary filesystem operations to CovertFS
\index{create() (File\_System.memfuse.MemFS method)}

\begin{fulllineitems}
\phantomsection\label{File_System:File_System.memfuse.MemFS.create}\pysiglinewithargsret{\bfcode{create}}{\emph{path}, \emph{mode}}{}
Makes a new file in mounted filesystem

\end{fulllineitems}

\index{flush() (File\_System.memfuse.MemFS method)}

\begin{fulllineitems}
\phantomsection\label{File_System:File_System.memfuse.MemFS.flush}\pysiglinewithargsret{\bfcode{flush}}{\emph{path}, \emph{fh}}{}
Writes all attributes and extra attributes of a file after it is done being accessed

\end{fulllineitems}

\index{fsync() (File\_System.memfuse.MemFS method)}

\begin{fulllineitems}
\phantomsection\label{File_System:File_System.memfuse.MemFS.fsync}\pysiglinewithargsret{\bfcode{fsync}}{\emph{path}, \emph{fdatasync}, \emph{fh}}{}
Same as flush

\end{fulllineitems}

\index{getattr() (File\_System.memfuse.MemFS method)}

\begin{fulllineitems}
\phantomsection\label{File_System:File_System.memfuse.MemFS.getattr}\pysiglinewithargsret{\bfcode{getattr}}{\emph{path}, \emph{fh=None}}{}
Function used extensively by the OS package FUSE for interaction with system calls

\end{fulllineitems}

\index{getxattr() (File\_System.memfuse.MemFS method)}

\begin{fulllineitems}
\phantomsection\label{File_System:File_System.memfuse.MemFS.getxattr}\pysiglinewithargsret{\bfcode{getxattr}}{\emph{path}, \emph{name}, \emph{position=0}}{}
Returns an additional attribute of a file (such as SELinux information)

\end{fulllineitems}

\index{listxattr() (File\_System.memfuse.MemFS method)}

\begin{fulllineitems}
\phantomsection\label{File_System:File_System.memfuse.MemFS.listxattr}\pysiglinewithargsret{\bfcode{listxattr}}{\emph{path}}{}
Lists extra attributes of a file

\end{fulllineitems}

\index{mkdir() (File\_System.memfuse.MemFS method)}

\begin{fulllineitems}
\phantomsection\label{File_System:File_System.memfuse.MemFS.mkdir}\pysiglinewithargsret{\bfcode{mkdir}}{\emph{path}, \emph{mode}}{}
Make a directory in mounted filesystem

\end{fulllineitems}

\index{open() (File\_System.memfuse.MemFS method)}

\begin{fulllineitems}
\phantomsection\label{File_System:File_System.memfuse.MemFS.open}\pysiglinewithargsret{\bfcode{open}}{\emph{path}, \emph{flags}}{}
Opens the file at a given path in mounted filesystem

\end{fulllineitems}

\index{read() (File\_System.memfuse.MemFS method)}

\begin{fulllineitems}
\phantomsection\label{File_System:File_System.memfuse.MemFS.read}\pysiglinewithargsret{\bfcode{read}}{\emph{path}, \emph{size}, \emph{offset}, \emph{fh}}{}
Reads the contents of a file and returns the requested number of bytes

\end{fulllineitems}

\index{readdir() (File\_System.memfuse.MemFS method)}

\begin{fulllineitems}
\phantomsection\label{File_System:File_System.memfuse.MemFS.readdir}\pysiglinewithargsret{\bfcode{readdir}}{\emph{path}, \emph{fh}}{}
Same as listing the contents of a directory

\end{fulllineitems}

\index{removexattr() (File\_System.memfuse.MemFS method)}

\begin{fulllineitems}
\phantomsection\label{File_System:File_System.memfuse.MemFS.removexattr}\pysiglinewithargsret{\bfcode{removexattr}}{\emph{path}, \emph{name}}{}
Removes the extra attribute from file

\end{fulllineitems}

\index{rename() (File\_System.memfuse.MemFS method)}

\begin{fulllineitems}
\phantomsection\label{File_System:File_System.memfuse.MemFS.rename}\pysiglinewithargsret{\bfcode{rename}}{\emph{old}, \emph{new}}{}
Rename a file or directory in a mounted filesystem

\end{fulllineitems}

\index{rmdir() (File\_System.memfuse.MemFS method)}

\begin{fulllineitems}
\phantomsection\label{File_System:File_System.memfuse.MemFS.rmdir}\pysiglinewithargsret{\bfcode{rmdir}}{\emph{path}}{}
Removes a directory from mounted filesystem

\end{fulllineitems}

\index{setxattr() (File\_System.memfuse.MemFS method)}

\begin{fulllineitems}
\phantomsection\label{File_System:File_System.memfuse.MemFS.setxattr}\pysiglinewithargsret{\bfcode{setxattr}}{\emph{path}, \emph{name}, \emph{value}, \emph{options}, \emph{position=0}}{}
Sets the extra attribute to given value

\end{fulllineitems}

\index{truncate() (File\_System.memfuse.MemFS method)}

\begin{fulllineitems}
\phantomsection\label{File_System:File_System.memfuse.MemFS.truncate}\pysiglinewithargsret{\bfcode{truncate}}{\emph{path}, \emph{length}, \emph{fh=None}}{}
Shortens the size of a file by a certain amount

\end{fulllineitems}

\index{unlink() (File\_System.memfuse.MemFS method)}

\begin{fulllineitems}
\phantomsection\label{File_System:File_System.memfuse.MemFS.unlink}\pysiglinewithargsret{\bfcode{unlink}}{\emph{path}}{}
Removes a file from the mounted filesystem

\end{fulllineitems}

\index{utimens() (File\_System.memfuse.MemFS method)}

\begin{fulllineitems}
\phantomsection\label{File_System:File_System.memfuse.MemFS.utimens}\pysiglinewithargsret{\bfcode{utimens}}{\emph{path}, \emph{times=None}}{}
Changes the modified and access times of a file in the mounted filesystem

\end{fulllineitems}

\index{write() (File\_System.memfuse.MemFS method)}

\begin{fulllineitems}
\phantomsection\label{File_System:File_System.memfuse.MemFS.write}\pysiglinewithargsret{\bfcode{write}}{\emph{path}, \emph{data}, \emph{offset}, \emph{fh}}{}
Writes the provided data to a file in the mounted filesystem

\end{fulllineitems}


\end{fulllineitems}

\index{mount() (in module File\_System.memfuse)}

\begin{fulllineitems}
\phantomsection\label{File_System:File_System.memfuse.mount}\pysiglinewithargsret{\code{File\_System.memfuse.}\bfcode{mount}}{\emph{memfs}, \emph{mountpoint}, \emph{db=False}}{}
Actually mounts the given memoryfs onto the given mountpoint

\end{fulllineitems}



\section{Image\_Manipulation package}
\label{Image_Manipulation:image-manipulation-package}\label{Image_Manipulation::doc}

\subsection{Submodules}
\label{Image_Manipulation:submodules}

\subsubsection{stegByteStream module}
\label{Image_Manipulation:stegbytestream-module}
The basic idea of the algorithm is to take each individual bit of the message and set it as the least significant bit of each component of each pixel of the image. Usually, a pixel has Red, Green, Blue components and sometimes an Alpha component. Because the values of these components change very little if the least significant bit is changed, the color difference is not particularly noticeable, if at all.
\phantomsection\label{Image_Manipulation:module-Image_Manipulation.lsbsteg}\index{Image\_Manipulation.lsbsteg (module)}\index{Steg (class in Image\_Manipulation.lsbsteg)}

\begin{fulllineitems}
\phantomsection\label{Image_Manipulation:Image_Manipulation.lsbsteg.Steg}\pysiglinewithargsret{\strong{class }\code{Image\_Manipulation.lsbsteg.}\bfcode{Steg}}{\emph{proxy}, \emph{online\_file\_store}}{}
Bases: \code{object}
\index{decode() (Image\_Manipulation.lsbsteg.Steg method)}

\begin{fulllineitems}
\phantomsection\label{Image_Manipulation:Image_Manipulation.lsbsteg.Steg.decode}\pysiglinewithargsret{\bfcode{decode}}{\emph{img}}{}
The decode method decodes the message from an image.
This method takes an image as a BytesIO object and returns a string
representing the bits of the message. To decode the bits:
\code{bytearray.fromhex('\%08X' \% int(message, 2))}.

\end{fulllineitems}

\index{decodeImageFromURL() (Image\_Manipulation.lsbsteg.Steg method)}

\begin{fulllineitems}
\phantomsection\label{Image_Manipulation:Image_Manipulation.lsbsteg.Steg.decodeImageFromURL}\pysiglinewithargsret{\bfcode{decodeImageFromURL}}{\emph{file\_id}}{}
The decodeImageFromURL method retrieves an image from a url,
and extracts a message from the image. The image needs to have
been encoded using the Steg.encode(msg) method.
This method takes a url as a string.
This method returns the decoded message as a bytearray.

\end{fulllineitems}

\index{encode() (Image\_Manipulation.lsbsteg.Steg method)}

\begin{fulllineitems}
\phantomsection\label{Image_Manipulation:Image_Manipulation.lsbsteg.Steg.encode}\pysiglinewithargsret{\bfcode{encode}}{\emph{message}}{}
The encode method encodes up to 1 byte of data per pixel.
This method takes a message as a bytearray object as a parameter.
This method returns a url as a string to the encoded message.

\end{fulllineitems}

\index{encodeSteg() (Image\_Manipulation.lsbsteg.Steg method)}

\begin{fulllineitems}
\phantomsection\label{Image_Manipulation:Image_Manipulation.lsbsteg.Steg.encodeSteg}\pysiglinewithargsret{\bfcode{encodeSteg}}{\emph{msg}, \emph{image}}{}
The encodeSteg method modifies the image and encodes the binary
message into the image using least significant bit stegenography.
This method takes a message as a byte array and an image as a
PIL Image object.
This method returns the url as a str.

\end{fulllineitems}


\end{fulllineitems}

\index{ascii2bits() (in module Image\_Manipulation.lsbsteg)}

\begin{fulllineitems}
\phantomsection\label{Image_Manipulation:Image_Manipulation.lsbsteg.ascii2bits}\pysiglinewithargsret{\code{Image\_Manipulation.lsbsteg.}\bfcode{ascii2bits}}{\emph{message}}{}
The ascii2bits function converts a string of ascii characters to a
padded string of bits.

\end{fulllineitems}



\subsubsection{genImage module}
\label{Image_Manipulation:genimage-module}
The \emph{genImage} module returns an image on request as a \emph{BytesIO} object.
\phantomsection\label{Image_Manipulation:module-Image_Manipulation.genImage}\index{Image\_Manipulation.genImage (module)}\index{genCatImage() (in module Image\_Manipulation.genImage)}

\begin{fulllineitems}
\phantomsection\label{Image_Manipulation:Image_Manipulation.genImage.genCatImage}\pysiglinewithargsret{\code{Image\_Manipulation.genImage.}\bfcode{genCatImage}}{}{}
The genCatImage function returns an image from The Cat API.
This function does not take any parameters.
This function returns a BytesIO object.

\end{fulllineitems}



\section{Web\_Connection package}
\label{Web_Connection:web-connection-package}\label{Web_Connection::doc}

\subsection{Submodules}
\label{Web_Connection:submodules}

\subsubsection{api\_cons module}
\label{Web_Connection:api-cons-module}
The \emph{api\_cons} module creates an anonymous connection to a given social media
file hosting website and provides connection, upload image, and download image
parameters.
\phantomsection\label{Web_Connection:module-Web_Connection.api_cons}\index{Web\_Connection.api\_cons (module)}\index{SendSpace (class in Web\_Connection.api\_cons)}

\begin{fulllineitems}
\phantomsection\label{Web_Connection:Web_Connection.api_cons.SendSpace}\pysiglinewithargsret{\strong{class }\code{Web\_Connection.api\_cons.}\bfcode{SendSpace}}{\emph{proxy}}{}
Bases: \code{object}

The SendSpace class is creates a connection to the SendSpace file
sharing website.
\index{connect() (Web\_Connection.api\_cons.SendSpace method)}

\begin{fulllineitems}
\phantomsection\label{Web_Connection:Web_Connection.api_cons.SendSpace.connect}\pysiglinewithargsret{\bfcode{connect}}{}{}
The connect method creates an anonymous connection to SendSpace.
The connection uses the sendspace API (1.1) and does not require
authentication or login.
This method requires no parameters.
This method returns the URL and the extra info needed for uploading an
image to SendSpace.

\end{fulllineitems}

\index{downloadImage() (Web\_Connection.api\_cons.SendSpace method)}

\begin{fulllineitems}
\phantomsection\label{Web_Connection:Web_Connection.api_cons.SendSpace.downloadImage}\pysiglinewithargsret{\bfcode{downloadImage}}{\emph{file\_id}}{}
The downloadImage method retrieves the direct download URL from the
download url returned by SendSpace.
This method take the download url returned by uploadImage as a
parameter.
This method returns the direct download url.

\end{fulllineitems}

\index{parseXML() (Web\_Connection.api\_cons.SendSpace method)}

\begin{fulllineitems}
\phantomsection\label{Web_Connection:Web_Connection.api_cons.SendSpace.parseXML}\pysiglinewithargsret{\bfcode{parseXML}}{\emph{xml}}{}
This method parses XML data with the BeautifulSoup module.

\end{fulllineitems}

\index{sendspace\_url (Web\_Connection.api\_cons.SendSpace attribute)}

\begin{fulllineitems}
\phantomsection\label{Web_Connection:Web_Connection.api_cons.SendSpace.sendspace_url}\pysigline{\bfcode{sendspace\_url}\strong{ = `http://api.sendspace.com/rest/'}}
\end{fulllineitems}

\index{upload() (Web\_Connection.api\_cons.SendSpace method)}

\begin{fulllineitems}
\phantomsection\label{Web_Connection:Web_Connection.api_cons.SendSpace.upload}\pysiglinewithargsret{\bfcode{upload}}{\emph{img}}{}
The upload method uploads an image to SendSpace.
This method takes an image as a BytesIO object.
This method returns the download and delete urls as a tuple.

\end{fulllineitems}

\index{uploadImage() (Web\_Connection.api\_cons.SendSpace method)}

\begin{fulllineitems}
\phantomsection\label{Web_Connection:Web_Connection.api_cons.SendSpace.uploadImage}\pysiglinewithargsret{\bfcode{uploadImage}}{\emph{upl\_url}, \emph{max\_size}, \emph{upl\_id}, \emph{upl\_extra\_info}, \emph{img}}{}
The uploadImage method sends an image file for upload to SendSpace.
This method requires the upload url and extra info from the SendSpace
connection and the image (BytesIO object) as parameters.
This method returns the direct download URL and delete URL of the image
as a tuple.

\end{fulllineitems}


\end{fulllineitems}



\subsubsection{proxy\_list module}
\label{Web_Connection:proxy-list-module}
The list of possible https and http proxies to use with \emph{sendspace}. Proxies are necessary on the DREN at USMA.
The DREN blocks many file-sharing websites, such as \emph{sendspace}.

Free US proxies:
\begin{itemize}
\item {} 
\code{http://165.139.149.169:3128}

\item {} 
\code{https://165.139.149.169:3128}

\end{itemize}


\renewcommand{\indexname}{Python Module Index}
\begin{theindex}
\def\bigletter#1{{\Large\sffamily#1}\nopagebreak\vspace{1mm}}
\bigletter{f}
\item {\texttt{File\_System.covertfs}}, \pageref{File_System:module-File_System.covertfs}
\item {\texttt{File\_System.memfuse}}, \pageref{File_System:module-File_System.memfuse}
\indexspace
\bigletter{i}
\item {\texttt{Image\_Manipulation.genImage}}, \pageref{Image_Manipulation:module-Image_Manipulation.genImage}
\item {\texttt{Image\_Manipulation.lsbsteg}}, \pageref{Image_Manipulation:module-Image_Manipulation.lsbsteg}
\indexspace
\bigletter{m}
\item {\texttt{main}}, \pageref{main:module-main}
\indexspace
\bigletter{w}
\item {\texttt{Web\_Connection.api\_cons}}, \pageref{Web_Connection:module-Web_Connection.api_cons}
\end{theindex}

\renewcommand{\indexname}{Index}
\printindex
\end{document}
